\documentclass[aspectratio=169,colorlinks]{beamer}
\usetheme{Boadilla} % plainest one with slide number footer

% generic packages

\usepackage[utf8]{inputenc}
\usepackage[english]{babel}
\usepackage{amsmath}
\usepackage{multicol}
\usepackage{ulem}

% about the presentation
\title[CoRAL]{The Constrained RESTful Application Language (CoRAL)}
\hypersetup{pdftitle={The Constrained RESTful Application Language (CoRAL)}}
\subtitle{\texttt{draft-ietf-core-coral-04}}
\author{\textit{Christian~Amsüss}, Thomas~Fossati}
\date{2021-11-08, IETF 112}

% used commands

\usepackage{verbatim}

\definecolor{darkgreen}{rgb}{0, 0.56, 0}

% attach self

\usepackage{embedfile}
\embedfile{\jobname.tex}

\begin{document}

\frame{\titlepage}

% recap, as this hasn't been in a full meeting since IETF108
\begin{frame}{CoRAL}\Large
	A data model and language for talking about resources and interactions with them,
	suitable for constrained devices
\end{frame}

\begin{frame}{Potential users}\Large
	\begin{itemize}
		\item \href{https://datatracker.ietf.org/doc/draft-ietf-core-problem-details/}{Problem details}
		\item \href{https://datatracker.ietf.org/doc/draft-ietf-core-coap-pubsub/}{PubSub} topic descriptions
		\item \href{https://datatracker.ietf.org/doc/draft-ietf-ace-oscore-gm-admin/}{OSCORE Group Manager administration}
		\item \href{https://datatracker.ietf.org/wg/asdf/documents/}{SDF}
		\item \ldots and anything that uses link-format (e.\,g. discovery)
	\end{itemize}
\end{frame}

\begin{frame}{In contrast to \ldots, CoRAL is:}\Large
	\begin{description}
		\setlength\itemsep{1em}
		\item[RDF] Compact (numeric pre-arranged or ad-hoc shorthands for predicates), parsable using CBOR, no URI processor required.

		\item[RFC6690] Less string parsing, more depth to information, clear semantics.
			% "No URI processor" could apply here too, but given the thigns done to URIs during 6690, it doesn't quite apply

		\item[CBOR] Semantic keys over ad-hoc ones. High-level terminology for derived specs. Interaction model provided. Reuse of terminology.
	\end{description}
\end{frame}

\begin{frame}{But CoRAL can be used with them:}\Large
	\begin{description}
		\setlength\itemsep{1em}
		\item[RDF] can be round-tripped to unstructured CoRAL almost completely.

		\item[RFC6690] can be round-tripped to CoRAL, provided the common CoRE attributes are used to describe the targets.

		\item[CBOR]\hspace{-0.3em}'s literals can be used in CoRAL.
	\end{description}
\end{frame}

\begin{frame}{Work areas}\large
	\begin{description}
		\item[90\%\footnote{Don't read too much into these numbers, they are for comparison between the items at best}]
			Information model

			Sea of triples, with optional structuring into a tree-like shape.

		\item[70\%] Interaction model

			User agent searches document, decides which link (or form) to follow.

		\item[70\%] Dictionary setup

			\href{https://datatracker.ietf.org/doc/draft-ietf-cbor-packed/}{Packed CBOR} now does the heavy lifting.
			Variations being discussed: Per-document-format; ad-hoc (Basic Packed); importing named dictionary.

			Document format can guide tree-like shape.

		\item[30\%] Binary serialization

			To be revisited with corpus of use case examples.

		\item[?] Text serialization

			Currently using binary serialization with \href{https://datatracker.ietf.org/doc/draft-bormann-cbor-edn-literals/}{EDN}
			(or \href{https://en.wikipedia.org/wiki/Turtle_(syntax)}{Turtle} when details like the optional structuring or which parts are compressed do not matter).

		\item[10\%] Queries, patches, provenance
	\end{description}
\end{frame}

\begin{frame}{Next steps}\Large
	\begin{itemize}
		\item Coordinate with users to validate current state against their models.
		\item Get corpus of examples for further dictionary and serialization work.
		\item What needs to be in for an initial usable version?
	\end{itemize}
\end{frame}

\begin{frame}{Thanks}\Large
	Comments?

	\bigskip

	Questions?

	\vspace{2cm}

	Design team for CoRAL and CRIs meets roughly every 2 weeks.
\end{frame}

\end{document}
